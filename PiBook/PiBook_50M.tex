\documentclass[12pt, oneside]{book} 
\usepackage{hyperref}

\usepackage{setspace} % Line spacing
\setstretch{1.5}

\usepackage{parskip} % Paragraph spacing
\setlength{\parskip}{1em} 

\usepackage{titlesec} % Customizing section and chapter titles
\titleformat{\maketitle}{\bfseries\huge}{\thetitle}{0pt}{\vspace{1em}}

\usepackage{fancyhdr} % Custom headers and footers
\pagestyle{fancy}
\fancyhf{} % Clear default header/footer
\fancyhead[C]{\textbf{$\pi$}} % Centering header content
\fancyfoot[C]{\thepage} % Centering page numbers on footer

\setlength{\parindent}{2em}  % Set first-line paragraph indentation

% For Chinese display 
\usepackage[AutoFallBack=true]{xeCJK} % !! Very important to enable AutoFallBack
\newfontlanguage{Chinese}{CHN}
\setCJKmainfont[Script=CJK,Language=Chinese]{SimSun}
\setCJKfallbackfamilyfont{rm}[Script=CJK,Language=Chinese]{PMingLiU-ExtB}


\title{《$\pi$》}
\author{}
\date{}
\begin{document}
	\maketitle
	
	{\centering
		\quad
		\vspace{4em}
		
		403.一本叫《$\pi$》的书 \\
		用四角编码,把$\pi$变成一串无尽的汉字,组成一本没有最后一页的书。\\
		2024.10.27 \\
		
		\vspace{1em}
		\href{https://mp.weixin.qq.com/s/LyUCRKytMIxOkjeeRd1OYg}{IDEA} \\ 
		by 孙智正 \\ 
		
		\vspace{2em}
		\href{https://github.com/yuanhes/4CornerIndex-to-Text}{IMPLEMENTATION} \\
		by 元和 \\ 
		
		\vspace{3em}
		PRODUCTION DATE \\
		\today
		
	}
	
	\pagenumbering{gobble} % Skipping page numbering before mainmatter
	
	\mainmatter
	
	\setcounter{page}{1} % Starting page numbering form 1
	
	% input the text 
	\input{text_from_pi_50M_digits.txt}
	
	......
	
\end{document}